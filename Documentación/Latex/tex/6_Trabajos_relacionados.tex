\capitulo{6}{Trabajos relacionados}

Existen multitud de trabajos y aplicaciones relacionadas con el mundo de la impresión 3D, pero casi todas ellas destinadas a gestionar una única impresora de manera similar a OctoPrint, como por ejemplo:

\section{AstroPrint}

AstroPrint \cite{astroprint} es una plataforma para controlar el uso de una impresora 3D desde la nube en cualquier tipo de dispositivo. Además crean su propio hardware desde el que poder controlar la impresora.

AstroPrint empezó su andadura como un \textit{fork} de OctoPrint, pero ahora sigue su propio desarrollo.

Actualmente existen algunas alternativas a OctoPrint las cuales las podemos consultar en la siguiente página web \cite{alternativas}. 

Otro trabajo relacionado con el control de una impresora 3D y que podría ser un complemento del presente trabajo es el desarrollado en la Universidad de Burgos por 
\textbf{Omar Santos}.

\section{3DPrinterMonitoring}

Este proyecto se encargaba de controlar cuando una impresión en curso era correcta respecto a un vídeo de referencia. De esta manera, mediante algoritmos de reconocimiento de imágenes, se podía detectar cuando la impresión era errónea y así poder cancelar la impresión con el fin de ahorrar tiempo y filamento.

Incluir una version de éste proyecto en nuestra aplicación web sería un gran complemento ya que el proyecto de Omar Santos se encargaría de detectar cuando la impresión no es correcta y con mi aplicación podemos cancelar la impresión en curso sin que el usuario tenga que hacerlo manualmente.