\capitulo{3}{Conceptos teóricos}

A continuación vamos a explicar algunos conceptos que son necesarios para comprender como funciona nuestra aplicación web.

\section{OctoPrint}

OctoPrint \cite{octoprint} es una interfaz web que sirve para controlar y monitorizar todos los aspectos que tienen que ver con las impresoras 3D desde el propio navegador.

OctoPrint está principalmente diseñado para funcionar sobre una Raspberry Pi conectada a una impresora 3D. El funcionamiento es sencillo, OctoPrint crea un servidor local desde el que se lanzan todos los servicios para controlar la máquina.

\imagen{octoprint}{Imagen correspondiente a una instancia de OctoPrint. Imagen extraída de \cite{wiki:octoprintimagen}.}

\section{API REST}

Para entender en que consiste la API \cite{api} de OctoPrint deberemos comprender qué es una API REST.

Una API REST es el conjunto de funciones que los desarrolladores podemos usar para obtener información de una aplicación web. Las solicitudes y las respuestas funcionan a través del protocolo HTTP. 

Las operaciones más utilizadas suelen ser GET y POST que son las que hemos usado en nuestra aplicación.

\section{JSON}

JSON \cite{wiki:json} es un formato de texto ligero para intercambio de datos.

Es el formato más utilizado cuando llevamos a cabo una comunicación con una API. Como por ejemplo, en nuestro caso cuando hacemos una petición GET a la API de OctoPrint ésta nos devuelve un archivo JSON en el que se encuentran los datos de las impresoras 3D.

\imagen{json}{Ejemplo de un archivo JSON en una petición GET. Imagen extraída de \cite{wiki:jsonimagen}.}

\section{G-code}

G-code \cite{wiki:gcode}, se conoce también como RS-274, es el nombre que recibe el lenguaje de programación más usado en control numérico.

Es usado principalmente en automatización y forma parte de la ingeniería asistida por computadora.

En el mundo de la impresión 3D se usa este lenguaje de programación para codificar los archivos que le enviamos a las impresoras 3D. Se podría decir que es un archivo con las coordenadas que debe seguir la impresora para realizar la impresión completa.


